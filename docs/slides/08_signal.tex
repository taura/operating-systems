\documentclass[12pt,dvipdfmx]{beamer}
\input{defs}
\title{シグナル}
\begin{document}
\maketitle

%%%%%%%%%%%%%%%%%%%%%%%%%%%%%%%%%%
\iffalse
\begin{frame}
\frametitle{目次}
\tableofcontents
\end{frame}
\fi

%%%%%%%%%%%%%%%%% 
%\section{シグナル}
%%%%%%%%%%%%%%%%% 

%%%%%%%%%%%%%%%%% 
\begin{frame}
  \frametitle{シグナルとは}
  \begin{itemize}
  \item イベント通知のためのAPI
  \item イベント通知 $\rightarrow$ シグナルの発生(配達)
    \begin{itemize}
    \item CPUに対する割り込みの「ソフトウェア(プロセス)版」
    \end{itemize}
  \item どんなイベントに対してシグナルが発生する?
    \begin{itemize}
    \item エラー, 例外的事象
      \begin{itemize}
      \item Segmentation Faultも実はその一つ
      \end{itemize}
    \item 時刻の経過など
    \item 明示的な送信(killシステムコール, killコマンド)
    \item etc.
    \end{itemize}
  \end{itemize}
\end{frame}

%%%%%%%%%%%%%%%%% 
\begin{frame}
  \frametitle{シグナルを受け取る方法}
  \begin{itemize}
  \item 発生したシグナルを受け取るいくつかの方法
    \begin{enumerate}
    \item 登録しておいた関数(シグナルハンドラ)が呼ばれる (sigaction)
    \item ファイルディスクリプタにデータが到着する (signalfd)
    \item シグナルの到着を待つ関数 (sigwaitinfo) に返り値が返される
    \end{enumerate}
  \item デフォルトの動作(何も登録されていない場合)はシグナルにより異なり,
    \begin{itemize}
    \item プロセスが強制終了される
    \item 無視される
    \end{itemize}
  \item ブロックする関数 (readなど) はシグナルを受け取るとリターンするものが多い
    (ブロックしていてもシグナルに気付けるように)
  \end{itemize}
\end{frame}

%%%%%%%%%%%%%%%%% 
\begin{frame}
  \frametitle{シグナルの例}
  \begin{itemize}
  \item []
    \ao{青字}が特によく使う・見かけるもの
    \begin{itemize}
    \item \ao{\tt SIGINT} : interrupt (典型的には端末でCtrl-Cを叩くと発生)
    \item {\tt SIGILL} : 不正命令の実行
    \item \ao{\tt SIGSEGV} : 不正なメモリのアクセス(Segmentation Fault)
    \item \ao{\tt SIGTERM} : プロセスの強制終了のためのシグナル(処理可能)
      \begin{itemize}
      \item {\tt kill}コマンドがデフォルトで送信するシグナル
      \end{itemize}
    \item \ao{\tt SIGKILL} :
      プロセスの強制終了のためのシグナル(処理不可能)
      \begin{itemize}
      \item {\tt kill -9/-KILL}が送信するシグナル
      \end{itemize}
    \item {\tt SIGALRM} :
      時間経過によって発生(alarm, setitimerシステムコール)
    \item {\tt SIGXCPU} :
      CPU使用量超過(処理不可能)
    \item {\tt SIGUSER1, 2} :
      明示的送信のための(自由に利用可能な)シグナル
    \end{itemize}
  \end{itemize}
\end{frame}

%%%%%%%%%%%%%%%%% 
\begin{frame}[fragile]
  \frametitle{シグナルの処理方法 (sigaction)}
  \begin{itemize}
  \item {\tt int sigaction({\it sig}, {\it act}, {\it oldact});}
    \begin{itemize}
    \item {\tt struct sigaction *} {\it act};
    \item シグナル{\it sig}を受信した時の動作を{\it act}で指定
    \end{itemize}
  \item {\tt struct sigaction}の中身
\begin{lstlisting}
struct sigaction {
  ...
  /* 呼ばれる関数を指定するフィールド */
  void (*@\ao{\tt sa\_sigaction}@)(int, siginfo_t *, void *); 
  sigset_t @\ao{\tt sa\_mask}@;
  int @\ao{\tt sa\_flags}@;
    ...
};
\end{lstlisting}
  \end{itemize}
\end{frame}

%%%%%%%%%%%%%%%%% 
\begin{frame}[fragile]
\frametitle{sigaction利用のテンプレート}
\begin{lstlisting}
/* シグナルハンドラ
   @{\it sig}@を受け取ったときに行う動作 */
void sigint_action(int sig, siginfo_t * info, void * arg) {
  ...
}


int main() { 
  ...

  /* @{\it sig}@ を受け取ったら sigint_actionを呼ぶように設定 */
  struct sigaction act;
  act.sa_sigaction = sigint_action;
  sigemptyset(&act.sa_mask);
  act.sa_flags = SA_SIGINFO;
  if (sigaction(@{\it sig}@, &act, 0) == -1) err(1, "sigaction");

  ...
}
\end{lstlisting}
\end{frame}

%%%%%%%%%%%%%%%%% 
\begin{frame}
  \frametitle{sigactionが設定されたシグナルが配達されたときの動作}
  \begin{itemize}
  \item あるスレッド(*)で, (ちょうど割り込みが起きたのと似たように)
    指定されたハンドラが実行される
  \item ハンドラが終了すると, 続きが実行される
  \item (*) あるスレッドはシグナルの種類により,
    \begin{itemize}
    \item そのプロセス中の(OSが適当に選んだ)1つのスレッド
    \item シグナルの原因となる命令を実行したスレッド(SIGSEGV, SIGILLなど)
    \end{itemize}
  \end{itemize}
\end{frame}

%%%%%%%%%%%%%%%%% 
\begin{frame}
  \frametitle{Segmentation Faultもシグナルの一つ}
  \begin{itemize}
  \item Segmentation Fault $=$ 不正なメモリアクセス時に発生 
    \begin{itemize}
    \item 割り当てられていないアドレスをアクセスした
    \item 割り当てられてはいるが, 保護属性(読み・書き・実行)で
      許可されていないアクセスが行われた
    \end{itemize}
  \item 関連システムコール
    \begin{itemize}
    \item {\tt mprotect(\ao{\it addr}, \ao{\it len}, \ao{\it prot});}
    \item {\tt mmap(\ao{\it addr}, \ao{\it len}, \ao{\it prot}, {\it flags}, {\it fd}, {\it offs});}
    \item {\it prot} --- {\tt PROT\_READ, PROT\_WRITE, PROT\_EXEC} (それらのbit和)
    \end{itemize}
  \item 例えば{\tt PROT\_WRITE}が設定されていない領域に書き込みを行うと,
    そのスレッドに{\tt SIGSEGV}シグナルが配達される
    
  \item シグナルハンドラを設定していなければ, プロセスが終了する
    (デフォルトのアクション)
  \end{itemize}
\end{frame}

\end{document}
