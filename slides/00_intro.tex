\documentclass[12pt,dvipdfmx]{beamer}
\input{defs}
\title{オペレーティングシステム --- イントロ}

\begin{document}
\maketitle

%%%%%%%%%%%%%%%%%%%%%%%%%%%%%%%%%% 
% \begin{frame}
% \frametitle{目次}
% \tableofcontents
% \end{frame}

%%%%%%%%%%%%%%%%% %%%%%%%%%%%%%%%%%
\section{}
%%%%%%%%%%%%%%%%% %%%%%%%%%%%%%%%%%

%%%%%%%%%%%%%%%%% 
\begin{frame}
\frametitle{オペレーティングシステム (Operating System; OS) とは}
\begin{itemize}
\item 実例: Windows, MacOS, Linux, BSD, iOS, Android, etc.
\item アプリケーションを動かすためのソフト(基本ソフト)
\item 存在理由(一般的な言葉で):
  \begin{itemize}
  \item \ao{抽象化:} 簡単にプログラムできるようにする
  \item \ao{効率化:} 簡単なプログラムで高速に動作するようにする
  \item \ao{資源保護・管理:} 資源(CPU, メモリ, etc.)の独占を防ぎ,
    公平に割り当てる
  \end{itemize}
\item 具体的には\mura{OSがないとどうなるか}を知る・考えるのが良い
\end{itemize}
\end{frame}

%%%%%%%%%%%%%%%%% 
\begin{frame}
  \frametitle{OSがないと\ldots}
  \begin{itemize}
  \item CPU (プロセッサ) 上に直接ユーザのプログラムが動く
  \item 例えば以下のようなことが非常に困難になる
    \begin{enumerate}
    \item \ao{CPU} (計算のための資源) を公平に分け合う
    \item \ao{メモリ} (記憶のための資源) を安全に分け合う
    \item \ao{外部ストレージ}を安全に分け合う
    \item \ao{入出力}
    \end{enumerate}
    etc.
  \end{itemize}
\end{frame}

%%%%%%%%%%%%%%%%% 
\begin{frame}
\frametitle{OSがない場合の問題点とOSの機能}
\begin{itemize}
\item CPUを分け合う
  \begin{itemize}
  \item OSなし: 1つのプログラムでCPUを独占できてしまう
  \item OS: \ao{プロセス, スレッド}
  \end{itemize}
\item メモリを分け合う
  \begin{itemize}
  \item OSなし: 1つのプログラムが他の人の
    (メモリ上の)データをのぞき見・破壊できてしまう;
    大量のメモリを独占できてしまう
  \item OS: \ao{プロセス(アドレス空間), 仮想記憶}
  \end{itemize}
\item ストレージを分け合う
  \begin{itemize}
  \item OSなし: 1つのプログラムが他の人のデータをのぞき見・破壊できてしまう
  \item OS: \ao{ファイルシステム}, \ao{システムコール}
  \end{itemize}
\item 入出力
  \begin{itemize}
  \item OSなし: 入力監視, 割り込み処理など複雑, かつ機器依存
  \item OS: \ao{ファイルシステム, プロセス間通信}
  \end{itemize}
\end{itemize}
\end{frame}


%%%%%%%%%%%%%%%%% 
\begin{frame}
  \frametitle{資源保護・管理のための基本的仕組み}
  \begin{itemize}
  \item 命題: 資源(CPU, メモリ, etc.)の独占を防ぎ, 公平に割り当てる
  \item 悪意のあるプログラムでも
    \mura{OSの破壊, 他のプログラムの破壊, 資源の独占}を不可能にする
  \item 以降ではまず\mura{OSの破壊}を不可能にする仕組みを考える
    \begin{itemize}
    \item それ以外は次週以降の個別の機能説明にて
    \end{itemize}
  \end{itemize}
\end{frame}

%%%%%%%%%%%%%%%%% 
\begin{frame}
\frametitle{OSは実体としてはどこにどう存在している?}
  \begin{columns}
    \begin{column}{0.6\textwidth}
      \begin{itemize}
      \item その他のプログラムと同様, メモリ上にプログラム$+$データとして存在
      \item ただし\ao{OS以外のプログラムには読み書き不能}になっている
      \item $\rightarrow$ どうやって?
      \end{itemize}
    \end{column}
    \begin{column}{0.4\textwidth}
      \includegraphics[width=\textwidth]{out/pdf/svg/os_1.pdf}
    \end{column}
  \end{columns}
\end{frame}

%%%%%%%%%%%%%%%%% 
\begin{frame}
\frametitle{CPUの特権モード・ユーザモード}
\begin{itemize}
\item CPUの動作モードに(少なくとも)2種類ある
  \begin{itemize}
  \item \ao{ユーザモード}
  \item \ao{特権モード (スーパバイザモード)}
  \end{itemize}
\end{itemize}
\begin{columns}
  \begin{column}{0.6\textwidth}
    \begin{itemize}
    \item 両者の主な違い
      \begin{enumerate}
      \item 一部の命令が特権モードでしか実行できない(\ao{特権命令})
      \item 一部のメモリ領域に「ユーザモードでアクセス不可」という属性をつけられる
      \end{enumerate}
    \item OSのデータやプログラムが
      \ao{OS以外のプログラムには読み書き不能}な仕組み
      \begin{itemize}
      \item OSが管理する領域を「ユーザモードでアクセス不可」
      \item OS以外はユーザモードで動作
      \end{itemize}
    \end{itemize}
  \end{column}
  \begin{column}{0.4\textwidth}
    \includegraphics[width=\textwidth]{out/pdf/svg/os_1.pdf}
  \end{column}
\end{columns}
\end{frame}

%%%%%%%%%%%%%%%%% 
\begin{frame}
\frametitle{ユーザモードから特権モードへの移行}
\begin{itemize}
\item<1-> 通常のプログラムはユーザモードで実行される
\item<1-> 一方通常のプログラムもOSの機能を呼び出すことが出来る
  (さもなければOSはいらないはず)
\item<2-> $\rightarrow$ ユーザモードから特権モードへ移行する仕組みがあるはず
\item<3-> 下手に設計すれば,
  結局誰でも特権モードで好きな命令を実行可能になる危険
\item<4-> $\rightarrow$ \ao{トラップ命令}
\end{itemize}
\end{frame}

%%%%%%%%%%%%%%%%% 
\begin{frame}
\frametitle{トラップ命令}
\begin{columns}
  \begin{column}{0.6\textwidth}
    \begin{itemize}
    \item 以下の二つを行う
      \begin{itemize}
      \item \ao{ユーザモードから特権モードへ移行}
      \item ある\ao{定められた番地}にジャンプ
      \end{itemize}
    \item x86の場合
      \begin{itemize}
      \item \href{https://www.felixcloutier.com/x86/syscall}{int 0x80h}命令
      \item \href{https://www.felixcloutier.com/x86/syscall}{syscall}命令
      \end{itemize}
    \item<2-> ある\ao{定められた番地}は,
      「割り込みベクタ」と呼ばれるメモリ上の
      配列にかかれており, OSが起動時に設定する
    \item<4-> ユーザプログラムからOSへの「入り口」
      $\rightarrow$ \ao{システムコール}
    \end{itemize}
  \end{column}
  \begin{column}{0.4\textwidth}
    \only<1>{\includegraphics[width=\textwidth]{out/pdf/svg/os_2.pdf}}%
    \only<2>{\includegraphics[width=\textwidth]{out/pdf/svg/os_3.pdf}}%
    \only<3>{\includegraphics[width=\textwidth]{out/pdf/svg/os_4.pdf}}%
    \only<4>{\includegraphics[width=\textwidth]{out/pdf/svg/os_5.pdf}}%
  \end{column}
\end{columns}
\end{frame}

%%%%%%%%%%%%%%%%% 
\begin{frame}
\frametitle{システムコール}
\begin{columns}
  \begin{column}{0.6\textwidth}
    \begin{itemize}
    \item OSがユーザに対して提供している(根源的な)機能
      \begin{itemize}
      \item 実例:
        open, write, read, close, fork, exec, wait, exit, socket, send, recv, etc.
      \item 通常Cの関数として説明されているがこれは説明の便宜上 $+$ ユーザの利便性のため
      \end{itemize}
    \item 本当にシステムコールが呼び出されている瞬間は,
      トラップ命令(前スライド)でOS内の命令に突入する瞬間
    \end{itemize}
  \end{column}
  \begin{column}{0.4\textwidth}
    \only<1>{\includegraphics[width=\textwidth]{out/pdf/svg/os_5.pdf}}%
    \only<2>{\includegraphics[width=\textwidth]{out/pdf/svg/os_6.pdf}}%
  \end{column}
\end{columns}
\end{frame}

%%%%%%%%%%%%%%%%% 
\begin{frame}
\frametitle{保護とシステムコール(すべての保護の基礎)}
\begin{columns}
  \begin{column}{0.6\textwidth}
    \begin{itemize}
    \item OS内には無数の機能が命令列として存在しているが,
      ユーザプログラムからの「入り口」
      (特権モードで実行される最初の命令)がひとつしかない
      
    \item その唯一の入り口から分岐してすべての機能ごとのシステムコールが実行されている
      
    \item ユーザプログラムが正規の入り口(システムコール)を通らずに,
      特権モードに移行することはできない
      
    \item OS内部の(特権モードで実行される)プログラムをしっかり書けば,
      OSを保護可能
    \end{itemize}
  \end{column}
  \begin{column}{0.4\textwidth}
    \includegraphics[width=\textwidth]{out/pdf/svg/os_6.pdf}
  \end{column}
\end{columns}
\end{frame}
\end{document}



